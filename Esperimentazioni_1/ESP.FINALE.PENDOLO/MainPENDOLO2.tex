\documentclass[a4paper,12pt]{article}

\usepackage[utf8]{inputenc}
\usepackage{graphicx}
\usepackage{amsmath}
\usepackage{amsfonts}
\usepackage{amssymb}
\usepackage{hyperref}
\usepackage{geometry}
\usepackage{caption}
\usepackage{subcaption}

\geometry{a4paper, margin=1in}

\captionsetup[table]{name=\textit{Tabella}}
\renewcommand*\contentsname{Indice}

\title{Relazione di laboratorio: Pendolo di Kater}
\author{Matteo Herz\\ Matricola: 1098162 \\ Università degli Studi di Torino}
\date{4 Giugno 2024}

\begin{document}
	
	\maketitle
	
	\begin{abstract}
		Questa relazione descrive gli esperimenti condotti e i risultati ottenuti nel laboratorio di [Nome del corso]. L'obiettivo dell'esperimento era [descrizione breve dell'obiettivo]. I risultati mostrano che [breve riassunto dei risultati principali].
	\end{abstract}
	
	\tableofcontents
	
	\section{Scopo dell'esperienza}
	L'obiettivo di questo esperimento è stato quello di [descrivere gli obiettivi]. L'importanza di questo esperimento risiede nel [spiegare il contesto e la rilevanza].
	
	\section{Strumentazione}
	\subsection{Materiali}
	Elenco dei materiali utilizzati:
	\begin{itemize}
		\item Materiale 1
		\item Materiale 2
		\item Materiale 3
	\end{itemize}
	
	\subsection{Metodi}
	Descrizione dettagliata dei metodi seguiti durante l'esperimento:
	\begin{enumerate}
		\item Primo passaggio
		\item Secondo passaggio
		\item Terzo passaggio
	\end{enumerate}
	
	\section{Presa dati}
	
	\section{Analisi dati}
	
	
	\begin{table}[h!]
		\centering
		\begin{tabular}{|c|c|c|}
			\hline
			Condizione & Valore 1 & Valore 2 \\
			\hline
			Condizione A & 10 & 20 \\
			\hline
			Condizione B & 30 & 40 \\
			\hline
		\end{tabular}
		\caption{Descrizione della tabella.}
		\label{tab:label}
	\end{table}
	
	
	\section{Conclusioni}
	Riassumere i principali risultati e la loro importanza. Proporre possibili sviluppi futuri per continuare il lavoro.
	
	\section{Riferimenti}
	\begin{thebibliography}{9}
		\bibitem{riferimento1}
		Autore 1, \textit{Titolo del libro o articolo}, Editore, Anno.
		\bibitem{riferimento2}
		Autore 2, \textit{Titolo del libro o articolo}, Editore, Anno.
	\end{thebibliography}
	
\end{document}
